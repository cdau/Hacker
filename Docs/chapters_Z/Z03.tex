\section{Aufgabe Z3}

Ein Experiment soll an mehreren aufeinanderfolgenden Tagen mit jeweils einer Maus durchgef"uhrt werden. Ihre Aufgabe ist es, zu Beginn festzulegen, an wie vielen Tagen dieses Experiment stattfinden soll. Es stehen 32 M"ause in einem K"afig zur Verf"ugung, von denen 24 Tr"ager eines bestimmten Merkmals (genetische Besonderheit, Krankheitssymptom, etc.) sind. Welche der M"ause dieses Merkmal tragen, ist nicht bekannt.

\begin{enumerate}[leftmargin=1cm, label=\alph*)]
	\item Zur Durchf"uhrung des Experiments muss die untersuchte Maus get"otet werden. Aus den 32 zur Verf"ugung stehenden M"ausen wird an jedem Versuchstag rein zuf"allig eine Maus ausgew"ahlt. Wie viele Tage m"ussen Sie f"ur die Versuchsreihe mindestens ansetzen, um mit einer Wahrscheinlichkeit von mindestens 0.9 mindestens 2 Versuche mit einer Maus durchzuf"uhren, die Tr"ager des Merkmals ist?
	
	\item In einem anderen Labor sind in derselben Situation die Untersuchungsmethoden so verbessert, dass die M"ause bei der Untersuchung nicht get"otet werden m"ussen. Dort wird an jedem Versuchstag zuf"allig eine Maus entnommen, die abends nach Beendigung des Experiments wieder in den K"afig zu ihren Artgenossen darf. Wie viele Tage muss man f"ur die Versuchsreihe mindestens ansetzen, um mit einer Wahrscheinlichkeit von mindestens 0.9 mindestens 2 Versuche mit einer Maus durchzuf"uhren, die Tr"ager des Merkmals ist?
\end{enumerate}

\newpage
Fortsetzung: