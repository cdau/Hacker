\section{Aufgabe E10}
In einer Bev�lkerungsgruppe werden Personen zuf�llig ausgew�hlt und ihre Cholesterinmenge im Blut ermittelt.
Es sei X das Alter der untersuchten Person, Y die Cholesterinmenge (in $mg/l$) im Blut und $Z$ gebe an, ob die Person Diabetiker ist ($Z = 1$, falls Diabetiker; $Z = 0$, sonst). Es sind folgende Wahrscheinlichkeiten bekannt:

\begin{onehalfspace}
\begin{tabbing}
\hspace*{4cm}\=\kill
\>$P(\{50 < X \le 60\}) = 0.13,$\\
\>$P(\{X \le 60\} \cap \{Z = 1\}) = 0.05,$\\
\>$P(\{X \le 60\} \cap \{Z = 0\}) = 0.75,$\\
\>$P(\{X \le 60\} \cap \{Y \le 150\} \cap \{Z = 1\}) = 0.01,$\\
\>$P(\{50 < X \le 60\} \cap \{Z = 0\}) = 0.01.$\\
\end{tabbing}
\end{onehalfspace}

\noindent Berechnen Sie die Wahrscheinlichkeiten der folgenden Ereignisse:
\begin{enumerate} [leftmargin=1cm, label=\alph*)]
	\item $\{X > 60\}$
	\item $\{X \le 50\}$
	\item $\{X \le 50\} \cap \{Z = 0\}$
	\item $\{X \le 60\} \cap \{Y > 150\} \cap \{Z = 1\}$
	\item Eine zuf�llig ausgew�hlte Person ist
	\begin{enumerate}[label=\roman*)]
	\item �lter als 60 Jahre oder Diabetiker
	\item h�chstens 50 Jahre alt, oder h�chstens 60 Jahre alt und nicht Diabetiker
\end{enumerate}
\end{enumerate}
