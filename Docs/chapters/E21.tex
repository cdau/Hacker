\section{Aufgabe E21}
Das Blut eines Patienten soll auf Antik�rper des Typs $\alpha$ untersucht werden. Zu diesem Zweck entnimmt der Arzt eine Blutprobe, teilt sie in drei gleichgro�e Teile und schickt diese an drei Labors. Die Testqualit�t der Labors ist gleich gut. Jedes Labor erh�lt bei Vorliegen von Antik�rpern des Typs $\alpha$ den Befund ?positiv? mit einer Wahrscheinlichkeit von $90\%$. Sind keine Antik�rper vorhanden, so ergibt sich in jedem Labor das Ergebnis ?negativ? mit einer Wahrscheinlichkeit von $80\%$.\vspace{10pt}
\newline
Nach einer Woche liegen die Analysen der drei Labors beim Arzt vor. Das Ergebnis lautet: Zwei positive
Befunde und ein negatives Testergebnis. Wie gro� ist die Wahrscheinlichkeit daf�r, dass im Blut des Patienten
Antik�rper des Typs $\alpha$ enthalten sind, wenn man davon ausgeht, dass allgemein $15\%$ aller Personen
solche Antik�rper besitzen.\vspace{10pt}
\newline
\textit{Hinweis}: Gehen Sie davon aus, dass die drei Labors unabh�ngig voneinander arbeiten.

