\section{Aufgabe E7}
Um den absoluten Nullpunkt der Temperatur zu bestimmen, kann man in einem eingeschlossenen Gas bei
konstantem Volumen die Temperatur und den Druck messen. Die nachfolgende Tabelle gibt die Messwerte
an. Es gilt: Wenn der Druck auf Null gesenkt wird, ist der absolute Nullpunkt der Temperatur erreicht.
\newline

\begin{center}
\begin{tabular}{|c|c|}
\hline  T:=Temperatur in Grad Celsius&  D:=Druck in mbar	\\ 
\hline  			25&						990.9		  	\\ 
\hline  			30&  					1008.5			\\ 
\hline  			35&  					1023.4			\\ 
\hline  			40&  					1035.6			\\ 
\hline  			50&  					1064.4			\\ 
\hline  			60&  					1095.2			\\ 
\hline  			70&  					1126.3			\\ 
\hline  			80&  					1153.4			\\ 
\hline  			85&  					1177.8			\\ 
\hline  			90&  					1207.6			\\ 
\hline  			93&  					1226.6			\\ 
\hline 
\end{tabular} 
\end{center}
Hilfe: Sie d"urfen verwenden, dass
\newline
\newline
$	\frac{1}{11}\sum _{i=1}^{11}{t}_{i}=59,8; 
	\hspace{1cm}
	\frac{1}{11}\sum _{i=1}^{11}({t}_{i})^2=4156,7;
	\hspace{1cm} 
	\frac{1}{11}\sum _{i=1}^{11}{d}_{i}=1100,8; 
	\newline
	\newline
	\frac{1}{11}\sum _{i=1}^{11}({d}_{i})^2=1218115; 
	\hspace{1cm}
	\frac{1}{11}\sum _{i=1}^{11}{t}_{i}*{d}_{i}=67742; 
$

\begin{enumerate}[leftmargin=1cm, label=\alph*)]
\item Bestimmen Sie den Korrelationskoeffizient.
\vspace{5cm}

\item Bestimmen Sie eine zur Frage passende Regressionsgerade und bestimmen Sie damit einen N"aherungswert f"ur die Temperatur am absoluten Nullpunkt in Grad Celsius.
\end{enumerate}

