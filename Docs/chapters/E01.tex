\section{Aufgabe E1}
Es soll eine neugez�chtete Kartoffelsorte getestet werden. Dazu wird ein Testfeld ausgew�hlt, das auf der gesamten Fl�che gleiche Wachstumsvoraussetzungen (Bodenqualit�t, Sonneneinstrahlung etc.) bietet. Nachdem eine gleichm��ige Pflege der Pflanzen erfolgte (Bew�sserung, D�ngung etc.), wurde von der ersten Ernte auf dem Versuchsfeld eine Stichprobe entnommen, die die folgende Urliste (f�r das Gewicht der Kartoffeln in g) lieferte:

\begin{center}
\renewcommand{\arraystretch}{1,5}
\begin{tabular}{|c|cccccccccc|}
\hline
$j$ & 1 & 2 & 3 & 4 & 5 & 6 & 7 & 8 & 9 & 10\\
\hline
$x_j$ & 132 & 145 & 172 & 151 & 152 & 136 & 143 & 112 & 159 & 152\\
\hline
\end{tabular}
\end{center}

\subsection{Bestimmen Sie die geordnete Stichprobe.}
	%Aufgabe 1a
	\vspace{0.5cm}
	\begin{center}
	\begin{tabular}{|c|cccccccccc|}
	\hline
	$j$ & 1 & 2 & 3 & 4 & 5 & 6 & 7 & 8 & 9 & 10\\
	\hline
	$x_j$ & 112 & 132 & 136 & 143 & 145 & 151 & 152 & 152 & 159 & 172\\
	\hline
	\end{tabular}
	\end{center}
	
\subsection{Klassieren Sie die obigen Daten (w�hlen Sie 4 gleichlange Intervalle) und zeichnen Sie das dazugeh�rige Histogramm.}
	

	\begin{tabular}{cc}
	 Intervalle&Anzahl\\
	 \hline
	 $I_{1}=[112,127[$&  $1$\\
	 $I_{2}=[127,142[$&  $2$\\ 
	 $I_{3}=[142,157[$&  $5$\\ 
	 $I_{4}=[157,172[$&  $2$\\ 
	\end{tabular} 
	\newline
	
	
	%Aufgabe 1c
\subsection{Skizzieren Sie die empirische Verteilungsfunktion der unklassierten Daten (F(x) = Anteil der Kartoffeln mit einem Gewicht kleiner gleich x).}
	\underline{L�sung:}
		\newpage
		
		
	%Aufgabe 1d
\subsection{Zeichnen Sie die empirische Verteilungsfunktion der klassierten Daten (wie bei unklassierten Daten au�er, dass innerhalb einer Klasse der Anteil als gleichm��ig wachsend betrachtet wird).}

		\underline{L�sung:}
		\vspace{8cm}
	
	
	%Aufgabe 1e
\subsection{Berechnen Sie den Mittelwert (=arithmetisches Mittel) der Daten.}
	\underline{L�sung:}
	\newline
	\newline
	$\overline{x}=\dfrac{1}{10}\sum_{n=0}^N x_{k}$
	
	
	%Aufgabe 1f
\subsection{Wie gross ist der Modalwert (der am h�ufigsten vorkommende Wert) der Stichprobe. Ist diese Information brauchbar?}
	\underline{L�sung:}
	\newline
	Modalwert = $152$
	\newline
	\newline
	
	
	%Aufgabe 1g
\subsection{Bestimmen Sie den Median (der Wert, so dass mind. 50\% der Daten kleiner gleich und mind. 50\% der Daten gr��er gleich diesem sind). Ist dieser Wert eindeutig?}
	\underline{L�sung:}
	\newline
	$x_{0,5}\in[x_{(5)},x_{(6)}]=[145,151]$
	\newline $x_{0,5}=\dfrac{x_{(5)}+x_{(6)}}{2}$
	\newline
	
	
	%Aufgabe 1h
\subsection{Bestimmen Sie das 0,25 Quantil (der Wert, so dass mind. 25\% der Daten kleiner gleich und mind. 75\% der Daten gr��er gleich diesem sind). Ist dieser Wert eindeutig?}

	\underline{L�sung:}
	\newline
	$x_{0,25}=x_{3}=136$
	\newline
	
	
	%Aufgabe 1i
\subsection{Bestimmen Sie das 0,75 Quantil (der Wert, so dass mind. 75\% der Daten kleiner gleich und mind. 25\% der Daten gr��er gleich diesem sind).}
	\underline{L�sung:}
	\newline
	$x_{0,75}=x_{(8)}=152$
	\newline
	
	
	%Aufgabe 1j
\subsection{Bestimmen Sie den Quartilsabstand (die Differenz zwischen dem 0,75 und dem 0,25 Quantil)}
	\underline{L�sung:}
	\newline
	Quartilsabstand $x_{0,75}-x_{0,25}=152-136=16$
	\newline
	
	
	
	%Aufgabe 1k
\subsection{Bestimmen Sie das 0,3 Quantil (der Wert, so dass mind. 30\% der Daten kleiner gleich und mind. 70\% der Daten gr��er gleich diesem sind). Ist dieser Wert eindeutig?}
	\underline{L�sung:}
	\newline
	$x_{0,3}\in[x_{(3)},x_{(5)}]$
	\newline
	
	%Aufgabe 1l
\subsection{Bestimmen Sie die empirische Varianz und daraus die empirische Standardabweichung.}
	\underline{L�sung:}
	\newline
	empirische Varianz $s=\dfrac{1}{n}\sum_{i=1}^n(x_{i}-\overline{x})$
	\newline
	empireische Standartabweichung $\sqrt{s^2}=\sqrt{\dfrac{1}{n}\sum_{i=1}^n(x_{i}-\overline{x}}$
	\newline
	\begin{center}
	
	\begin{tabular}{|c|c|c|}
	\hline \rule[-2ex]{0pt}{5.5ex}  Index&  Gewicht&  $x(i)-\overline{x}$\\ 
	\hline \rule[-2ex]{0pt}{5.5ex}  1&  		112&  	$[112-145,4]^2$\\ 
	\hline \rule[-2ex]{0pt}{5.5ex}  2&  		132&  	$[132-145,4]^2$\\ 
	\hline \rule[-2ex]{0pt}{5.5ex}  3&  		136&  	$[136-145,4]^2$\\ 
	\hline \rule[-2ex]{0pt}{5.5ex}  4&  		143&  	$[143-145,4]^2$\\ 
	\hline \rule[-2ex]{0pt}{5.5ex}  5&  		145&  	$[145-145,4]^2$\\ 
	\hline \rule[-2ex]{0pt}{5.5ex}  6&  		151&  	$[151-145,4]^2$\\ 
	\hline \rule[-2ex]{0pt}{5.5ex}  7&  		152&  	$[152-145,4]^2$\\ 
	\hline \rule[-2ex]{0pt}{5.5ex}  8&  		152&  	$[152-145,4]^2$\\ 
	\hline \rule[-2ex]{0pt}{5.5ex}  9&  		159&  	$[159-145,4]^2$\\ 
	\hline \rule[-2ex]{0pt}{5.5ex}  10&  		172&  	$[172-145,4]^2$\\ 
	\hline \rule[-2ex]{0pt}{5.5ex}  Summe&  	1454&  	-\\ 
	\hline \rule[-2ex]{0pt}{5.5ex}  Mittelwert& 145,4&  -\\ 
	\hline 
	\end{tabular} 
	\end{center}
	
	\begin{center}
		
	\begin{tabular}{|c|c|c|}
	\hline \rule[-2ex]{0pt}{5.5ex}  Index&  Gewicht&  $x(i)-\overline{x}$\\ 
	\hline \rule[-2ex]{0pt}{5.5ex}  1&  		112&  	$1115,56$\\ 
	\hline \rule[-2ex]{0pt}{5.5ex}  2&  		132&  	$179,56$\\ 
	\hline \rule[-2ex]{0pt}{5.5ex}  3&  		136&  	$88,36$\\
	\hline \rule[-2ex]{0pt}{5.5ex}  4&  		143&  	$5,76$\\ 
	\hline \rule[-2ex]{0pt}{5.5ex}  5&  		145&  	$0,16$\\ 
	\hline \rule[-2ex]{0pt}{5.5ex}  6&  		151&  	$31,36$\\ 
	\hline \rule[-2ex]{0pt}{5.5ex}  7&  		152&  	$43,56$\\
	\hline \rule[-2ex]{0pt}{5.5ex}  8&  		152&  	$43,56$\\
	\hline \rule[-2ex]{0pt}{5.5ex}  9&  		159&  	$182,96$\\
	\hline \rule[-2ex]{0pt}{5.5ex}  10&  		172&  	$707,56$\\
	\hline \rule[-2ex]{0pt}{5.5ex}  -&  		Summe&  $2400,4$\\ 
	\hline \rule[-2ex]{0pt}{5.5ex}  -& 			$s^2$&  $240,04$\\ 
	\hline \rule[-2ex]{0pt}{5.5ex}  -& 			$s$&  	$15,49$\\
	\hline 
	\end{tabular} 
	\end{center}
	
	%Aufgabe 1m
\subsection{Zeichnen Sie den zu dieser Verteilung (unklassierte Daten) passende Boxplot.}
	\underline{L�sung:}
	\newline
	Minimum $x_{min} =112$\newline
	Maximum  $x_{max} =172$\newline
	unteres Quantil $x_{0,25}=136$\newline
	unteres Quantil $x_{0,75}=152$\newline
	Median $x_{0,5}=148$\newline
	arithmetisches Mittel $\overline{x}=1145,4$\newline
	\newline
	\newline
	Grafik fehlt
	\newline
	
	%Aufgabe 1n
\subsection{Wenn man den Wert 136 durch 316 ersetzt (Schreibfehler), wie wirkt das dann auf den Mittelwert, die Standardabweichung, den Median und die Quantile aus (nur qualitativ: also w�chst/f�llt leicht/stark)?}
	\underline{L�sung:}
	\newline
	$136 \rightarrow 316$
	$x_{(0,5)}$ w�chst leicht 				\newline
	Quantile wachsen langsam oder nicht		\newline
	Quartilsabstand w�chst leicht			\newline
	$\overline{x}$ w�chst stark				\newline
	s sehr stark							\newline
	\newline
	