\subsection*{Aufgabe 2}

Betrachten Sie die folgende reelle Funktion F (mit vorgegebenen reellen Zahlen a < b < c):\newline

 F(x)=$\begin{cases}
		0$ \hspace{60pt}f�r  x < a $\\
		\frac{1}{2} \cdot \frac{x-a}{b-a} $ \hspace{29pt} f�r $a\le x\le b$ $\\
		\frac{1}{2} + \frac{1}{2} \cdot \frac{x-b}{c-b}$ \hspace{10pt} f�r $b\le x\le c$ $\\
		1$ \hspace{57pt} f�r x > c $\\
		
		\end{cases}$

\begin{enumerate}[leftmargin=0.7cm, label=\alph*)]
\item Warum handelt es sich hier um eine Verteilungsfunktion? Ist es die Verteilungsfunktion
einer diskreten oder einer stetigen Zufallsvariablen?
\vspace{4cm}

\item Bestimmen Sie die zugeh�rige Dichtefunktion.
\vspace{4cm}

\item Wie lautet das 20\%-Quantil der Verteilung?
\vspace{4cm}

\item Wie lautet der Erwartungswert der Verteilung?
\end{enumerate}