\subsection*{Aufgabe 12)}
(Klausur WS 2011/12) Um mit einer Geldkarte am Geldautomaten Bargeld abzuheben, ist die Eingabe der so genannten PIN (Pers�nliche Identifikationsnummer) erforderlich. Eine PIN
besteht aus vier Ziffern zwischen 0 und 9 (die beliebig oft vorkommen k�nnen).

\begin{enumerate} [leftmargin=0.7cm, label=\alph*)]
\item �blicherweise hat man zur Eingabe der PIN an einem Geldautomaten nur eine bestimmte Zahl falscher Versuche, bevor die Karte einbehalten wird.
\newline \newline
Nehmen Sie an, jemand findet Ihre Geldkarte und probiert auf gut Gl�ck (also: Laplace-Experiment) einige unterschiedliche PIN-Nummern an einem Geldautomaten aus. Mit welcher Wahrscheinlichkeit kann diese Person Ihr Konto anzapfen, wenn nach drei falschen PIN-Eingaben die Karte einbehalten wird?
\vspace{4cm}
\item Wie �ndert sich die Wahrscheinlichkeit aus a), wenn die Person wei�, dass Ihre PINNummer aus lauter verschiedenen Ziffern besteht?
\vspace{4cm}
\item Wie gro� w�re die Wahrscheinlichkeit aus a), wenn die Person aus irgendeinem Grund wei�, dass in Ihrer PIN genau zweimal die Ziffer 8 vorkommt?
\end{enumerate}
