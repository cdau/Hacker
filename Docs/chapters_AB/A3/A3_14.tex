\subsection*{Aufgabe 14}
(Klausur WS 2013/14) Es soll ein bin�res Wort, das aus 4 Bit besteht, �ber einen Kommunikationskanal �bertragen werden. Dazu werden die einzelnen Bit nacheinander verschickt und in derselben Reihenfolge empfangen. Allerdings kann bei der �bertragung eines Bit ein Fehler passieren (so dass eine gesendete 0 als 1 empfangen wird oder umgekehrt). Die Wahrscheinlichkeit f�r einen Fehler bei der �bertragung eines Bit sei p; Fehler bei der �bertragung verschiedener Bit passieren unabh�ngig voneinander.

\begin{enumerate} [leftmargin=0.6cm, label=\alph*)]
\item Mit welcher Wahrscheinlichkeit (in Abh�ngigkeit von p) wird ein aus vier Bit bestehendes Wort korrekt �bertragen?
\item Um die Wahrscheinlichkeit einer erfolgreichen �bertragung zu erh�hen, kann ein 7-Bit Hamming-Code verwendet werden. Details sind hier uninteressant, es gen�gt zu wissen, dass ein 4-Bit-Wort dabei in ein 7-Bit-Wort (also ein aus 7 Bit bestehendes Wort) umgeformt und dieses dann verschickt wird. Aus dem empfangenen 7-Bit-Wort kann das urspr�ngliche 4-Bit-Wort korrekt rekonstruiert werden, wenn bei der �bertragung des 7-Bit-Worts bei h�chstens einem Bit ein Fehler passiert. Mit welcher Wahrscheinlichkeit (in Abh�ngigkeit von p) l�sst sich mit dieser Methode ein urspr�nglich 4 Bit langes Wort korrekt �ber den Kommunikationskanal verschicken?
\end{enumerate}
