\subsection*{Aufgabe 10}
�ber einen bin�ren Kommunikationskanal werden die Signale ?0? und ?1? �bertragen. Aufgrund eines Rauschens werden die Signale manchmal verf�lscht, d.h. eine gesendete ?0? wird manchmal als ?1? empfangen und umgekehrt. Eine ?0? wird mit einer Wahrscheinlichkeit von 0,94, eine ?1? mit einer Wahrscheinlichkeit
von 0,91 korrekt �bertragen. 55\% der gesendeten Zeichen sind eine ?1?.
\begin{enumerate} [leftmargin=0.7cm, label=\alph*)]
\item Mit welcher Wahrscheinlichkeit wird bei �bertragung eines Zeichens eine ?0? empfangen? Mit welcher Wahrscheinlichkeit wird eine ?1? empfangen?
\item Bestimmen Sie die Wahrscheinlichkeit daf�r, dass eine ?1? gesendet wurde, wenn eine ?0? empfangen wurde.
\item Welcher Anteil der empfangenen Zeichen wurde anders verschickt als empfangen?
\end{enumerate}
L�sen die Aufgabe auf zwei Arten: Mit Hilfe der Wahrscheinlichkeitstheorie und alternativ durch Aufstellen einer Kreuztabelle.