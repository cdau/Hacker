\subsection*{Aufgabe 4}
Ein paar kurze (!) Kombinatorik-Aufgaben; Angaben der L�sungsformeln reicht, es ist keine Berechnung n�tig
\begin{enumerate}[leftmargin=0.5cm, label=\alph*)]
\item Auf ein Schachbrett (mit 64 Feldern, davon je die H�lfte schwarz und wei�) werden 6
identische Spielsteine auf 6 verschiedene Felder gesetzt. Auf wie viele Arten ist dies
m�glich?
\item Auf einem Schachbrett werden 6 verschiedene Felder ausgew�hlt. Mit welcher
Wahrscheinlichkeit sind alle Felder schwarz?
\item Mit welcher Wahrscheinlichkeit erzielt ein Spieler beim Lotto mit einem Tipp
mindestens 4 Richtige?
\item Auf einem Schachbrett werden 3 verschiedene Felder ausgew�hlt. Mit welcher
Wahrscheinlichkeit sind die Felder nicht alle von der gleichen Farbe?
\item Eine Gruppe besteht aus 10 Mathematikern, 12 Informatikern und 15 Soziologen. Nun
sollen drei verschiedene Aufgaben vergeben werden; je eine Aufgabe soll von einem
Mathematiker, einem Informatiker und einem Soziologen durchgef�hrt werden. Auf wie
viele Arten k�nnen die drei Aufgaben verteilt werden?
\item Im Dunkeln werden aus einer Schublade mit 10 schwarzen und 16 blauen Socken zuf�llig
zwei Socken entnommen. Mit welcher Wahrscheinlichkeit haben die beiden Socken
dieselbe Farbe?
\item 5 W�rfel werden gleichzeitig geworfen. Mit welcher Wahrscheinlichkeit werden nur
verschiedene Zahlen geworfen?
\item 7 Personen spielen zusammen Badminton. Zur Verf�gung steht ein Spielfeld; es soll
Einzel gespielt werden. Wie viele Spiele finden statt, wenn jeder genau einmal gegen
jeden anderen spielen will?
\item 7 Personen spielen zusammen Badminton. Zur Verf�gung steht ein Spielfeld; es soll
Doppel gespielt werden. Wie viele Spiele finden statt, wenn jedes m�gliche Paar genau
einmal gegen jedes andere m�gliche Paar spielen will? (Achtung: etwas schwieriger.)
\item 32 Spielkarten (darunter 4 Damen) werden auf 4 Spieler (darunter Spieler A) verteilt
(jeder erh�lt 8 Karten). Mit welcher Wahrscheinlichkeit erh�lt Spieler A keine Dame?
\item 32 Spielkarten (darunter 4 Damen) werden auf 4 Spieler (darunter Spieler A) verteilt
(jeder erh�lt 8 Karten). Mit welcher Wahrscheinlichkeit erh�lt Spieler A mindestens eine
Dame?
\item Wie viele ganze Zahlen gibt es, die in Dualdarstellung h�chstens 10 Ziffern umfassen, wobei
darin genau sechsmal die Ziffer 1 vorkommt?
\item Wie viele 20-stellige nat�rliche Zahlen gibt es, in deren Ziffernfolge (in
Dezimalschreibweise) genau achtmal die 1, viermal die 2 und f�nfmal die 3 vorkommt
und zus�tzlich drei verschiedene andere Ziffern ungleich 0 je einmal vorkommen?
\item Wie viele Stellen werden in bin�rer Darstellung mindestens ben�tigt, um die ganzen
Zahlen von 0 bis 999999999 als Dualzahlen darzustellen? 
\end{enumerate}
