\subsection*{Aufgabe 3}
F�r die Messung von (wirtschaftlicher) Armut gibt es mehrere Definitionen. H�ufig wird ein
Haushalt (vereinfacht) als arm definiert, wenn das Haushaltseinkommen weniger als 60\% des
Medianhaushaltseinkommens betr�gt. Alternativ k�nnte man einen Haushalt als arm
definieren, wenn das Haushaltseinkommen weniger als 60\% des arithmetischen Mittels aller
Haushaltseinkommen betr�gt. Erl�utern Sie, nach welcher dieser beiden Armutsdefinitionen
in Deutschland mehr Haushalte als arm bezeichnet w�rden!
