\subsection*{Aufgabe 12}
Ein metrisches Merkmal X wird in zwei verschiedenen Teilgruppen beobachtet. In beiden
Teilgruppen wurden Minimum, Maximum, Median und arithmetisches Mittel berechnet:
\begin{center}
\begin{tabular}{|c|c|c|c|c|c|}
\hline Minimum & Maximum & Median & arithmetisches Mittel & Gr��e der Teilgruppe\\
\hline 0 & 4 & 3 & 2 & 20 \\
\hline 3 & 9 & 5 & 6 & 30 \\
\hline
\end{tabular}
\end{center}
Berechnen Sie, falls m�glich, aus den angegebenen Werten das gesamte arithmetische Mittel,
den gesamten Median, die gesamte Spannweite und die gesamte Varianz von X, wenn die 50
betrachteten Merkmalstr�ger als Gesamtheit angesehen werden. Falls Ihnen die Berechnung
einiger oder aller Ma�zahlen nicht m�glich ist, geben Sie an, welche zus�tzlichen
Informationen Sie dazu jeweils br�uchten. Geben Sie f�r alle Ma�zahlen, die Sie nicht
berechnen k�nnen, m�glichst kleine Intervalle an, in denen die Ma�zahlen mit Sicherheit
liegen.

