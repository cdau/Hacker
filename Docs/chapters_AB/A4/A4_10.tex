\subsection*{Aufgabe 10}
Bei einem h�ufig durchgef�hrten Routinetest auf eine bestimmte Krankheit wird eine 
Blutprobe auf die Anwesenheit von bestimmten Antik�rpern getestet. Da es sich um eine 
seltene Krankheit handelt (der Anteil der Kranken in der Bev�lkerung sei p) und der Test 
daher fast immer negativ ausf�llt, wird folgendes Vorgehen vorgeschlagen (so genanntes 
Gruppen-Screening): Nicht jede Blutprobe wird einzeln untersucht, sondern es werden immer bestimmte Mengen aus n verschiedenen  Blutproben zusammengesch�ttet und gut 
durchmischt. Anschlie�end wird ein Test mit dieser neuen (Misch-)Probe gemacht. F�llt der Test negativ aus, folgt damit, dass alle n Blutproben negativ sind. In diesem Fall hat man n?1 Tests eingespart. F�llt der Test mit der gemischten Probe hingegen positiv aus, werden anschlie�end alle n urspr�nglichen Proben einzeln getestet; in diesem Fall braucht man insgesamt n + 1 Tests. Bestimmen Sie den Erwartungswert der n�tigen Tests beim GruppenScreening von n Proben in Abh�ngigkeit von p und n. F�r welche Werte von p lohnt sich das Gruppenscreening, wenn n = 20 gesetzt wird?