\subsection*{Aufgabe 12}
(Klausur WS 2013/14)  Vor zwei Jahren wurden anl�sslich der Fu�ball-Europameisterschaft 
wieder Panini-Sammelbilder verkauft. Insgesamt gab es 540 verschiedene Bilder (nummeriert von 1 bis 540); die Bilder wurden nicht einzeln, sondern in (blickdichten) P�ckchen mit je 5 verschiedenen (!) Bildern verkauft. (Welche Bilder in einem P�ckchen sind, wusste man nat�rlich erst, wenn man es ge�ffnet hat.) Nehmen Sie an, dass die Inhalte verschiedener P�ckchen voneinander unabh�ngig sind, und dass jede m�gliche Kombination aus 5 verschiedenen Bildern mit gleicher Wahrscheinlichkeit in einem P�ckchen zu finden ist. Geben Sie f�r die nachfolgenden Zufallsvariablen an, welche Verteilung sie besitzen (Name und konkrete Parameter der Verteilung reichen aus), und berechnen Sie jeweils den Erwartungswert.

\begin{enumerate} [label=\alph*)]
\item Zahl der P�ckchen, die man kaufen muss, bis man das Bild mit der Nummer 1 erh�lt.
\item Zahl der Bilder mit der Nummer 1, die man beim Kauf von 270 P�ckchen erhalten wird.
\end{enumerate}